\def\TIMER555(#1)#2{%
  \begin{scope}[shift={(#1)}]
    \draw[fill=blue!10] (-1.5,-2) rectangle (1.5,2); % The body of IC
    % Label and component identifier.
    \draw[blue] (2,2.5) node []{\large \bf IC - #2}; % IC LABEL
    \draw[blue] (0,0.5) node [align=center]{\large NE-555\\TIMER}; % IC LABEL
    % Draw the pins
    % Some that you have to learn about label nodes, draw lines, and name coordinates in Tikz
    \draw (0.9,-2) node [above]{GND} -- +(0,-0.5) node [anchor=-45]{1} coordinate (#2 GND); % Pin 1 GND
    \draw (-1.5,-1.5) node [right]{TRG} -- +(-0.5,0) node [anchor=-135]{2} coordinate (#2 TRG); % Pin 2 TRG
    \draw (1.5,0) node [left]{OUT} -- +(0.5,0) node [anchor=-45]{3} coordinate (#2 OUT); % Pin 3 OUT  
    \draw (0.9,2) node [below]{RESET} -- +(0,0.5) node [anchor=45]{4} coordinate (#2 RESET); % Pin 4 RESET
    \draw (0,-2) node [above]{CTRL} -- +(0,-0.5) node [anchor=-45]{5} coordinate (#2 CTRL); % Pin 5 CTRL
    \draw (-1.5,-.5) node [right]{THR} -- +(-0.5,0) node [anchor=-135]{6} coordinate (#2 THR); % Pin 6 THR
    \draw (-1.5,1.5) node [right]{DIS} -- +(-0.5,0) node [anchor=-135]{7} coordinate (#2 DIS); % Pin 7 DIS
    \draw (0,2) node [below]{$\mathsf{V_{CC}}$} -- +(0,0.5) node [anchor=45]{8} coordinate (#2 VCC); % Pin 8 VCC
  \end{scope}
}

\tikzset{flipflop sr-rst/.style={flipflop,
    flipflop def={t1=S, t3=R, t6=Q, n6=1, nu=1},
}}