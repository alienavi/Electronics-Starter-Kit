%----------------------------------------------------------------------------------------
%	CHAPTER 4
%----------------------------------------------------------------------------------------

\chapterimage{head1.png} % Chapter heading image

\chapter{555}

\section{Overview}
In this section you'll learn about one of the most famous integrated circuit (IC) in use. Each year millions of 555 Timer ICs are manufactured and sold.
It's named 555 because there are three \SI{5}{\kilo\ohm} resistors inside the IC. And as the name suggest, it is a timer circuit. The timing interval is controlled by an external resistor/capacitor network. And by changing the values for the resistor and capacitor the timing duration can be easily varied.

Let's take a look at the pins of 555 Timer IC:
\begin{enumerate}
    \item \textbf{GND - Pin 1} Ground pin of the IC
    \item \textbf{VCC - Pin 8} Positive supply is connected to this pin, the voltage must be at least \SI{4.5}{\volt} and maximum \SI{15}{\volt}.
    \item \textbf{OUT - Pin 3} The output is either low (close to \SI{0}{\volt}) or high (close to VCC).
    \item \textbf{TRG - Pin 2} Trigger is active low, which means when the voltage on this pin drops below one-third of the supply voltage, the output of 555 goes high.
    \item \textbf{DIS - Pin 7} This pin is used to discharge an external capacitor that works in conjunction with a resistor to control the timing of the 555 IC.
    \item \textbf{THR - Pin 6} Threshold pin is used to monitor the voltage across the capacitor that's discharged by pin 7. When this voltage reaches two-third of the supply voltage, the output goes low.
    \item \textbf{CTRL - Pin 5} Control pin can be used to vary the voltage level at the inverting input of the threshold comparator. It is generally connected to ground via \SI{0.01}{\micro\farad} capacitor to eliminate any fluctuation on noise in the operation of the timer.
    \item \textbf{RST - Pin 4} Reset pin is active low, which means when this pin is momentarily grounded the 555 timer will reset it's state and will stop until it is triggered again.
\end{enumerate}

\begin{figure}[htp]
    \centering
    \begin{circuitikz}[scale = 1.5]
        \draw (0,0) to [short, *-] (4.5,0) {};
        \draw (4.5,0) node[npn, anchor=E, rotate=-90, color=red](npn1){};
        \draw (npn1.C) to [short, -o] (7,0){} node[above=1mm]{DIS} node[below=1mm]{7};
        \draw (0,0) node[ground]{};
        \draw (0,0) 
            to [R, l=\SI{5}{\kilo\ohm}, color=blue] (0,1.8)
            to [R, l=\SI{5}{\kilo\ohm}, color=blue] (0, 3.6)
            to [R, l=\SI{5}{\kilo\ohm}, color=blue] (0, 5.4);
        \draw (-0.2,5.4) -- node[anchor=south, color=red] {VCC} (0.2,5.4);
        \draw (1.2,1.8) node[op amp, color=red](opamp1){};
        \draw (1.2,3.6) node[op amp, color=red](opamp2){};
        \draw 
            (opamp1.+) to [short, -*] ++(-0.4,0){}
            (opamp2.-) to [short, -*] ++(-0.4,0){} 
                to [short, -o] ++(-1,0)
                node[above=0.2mm]{CTRL} node[below=0.2mm]{5};
        \draw
            (opamp1.-) to [crossing] ++(-0.8,0) to[short, -o] ++(-0.6,0) 
            node[above=0.2mm]{TRG} node[below=0.2mm]{2}
            (opamp2.+) to [crossing] ++(-0.8,0) to[short, -o] ++(-0.6,0)
            node[above=0.2mm]{THR} node[below=0.2mm]{6};
        \draw (3.5,2.7) node[flipflop sr-rst, external pins width=0] (SR1){};
        \draw 
            (SR1.pin 1) -| (opamp2.out)
            (SR1.pin 3) -| (opamp1.out)
            (SR1.up) -- ++(0,1) 
            to [short,-o] ++(0,1) 
            node[left=1mm]{\ctikztextnot{RST}} node[right=1mm]{4};
        \draw 
            let 
                \p1 = (npn1.B),
                \p2 = (SR1.pin 6)
            in 
                (SR1.pin 6) to [short, -*] (\x1, \y2){}
                (npn1.B) -- (\x1, \y2){}
                (\x1, \y2) to[inline not] ++(2,0) 
                to[short, -o] ++(0,0) node[above=1mm]{OUT} node[below=1mm]{3};
        %\draw (SR1.pin 6) to [short, -*] ++(1,0) coordinate(nin);
        %\draw (npn1.B) -- (nin);
        %\draw (nin) to[inline not] ++(2,0);
    \end{circuitikz}
    \caption{555 Timer Circuit}
    \label{fig:555_internal_circuit}
\end{figure}

Figure \ref{fig:555_symbol} shows the schematic symbol for 555 IC that we will use in this chapter's circuit examples.
\begin{figure}[htp]
    \centering
    \begin{circuitikz}[scale = 1.5]
        \TIMER555(0,0){1}
    \end{circuitikz}
    \caption{555 Timer Symbol}
    \label{fig:555_symbol}
\end{figure}