%----------------------------------------------------------------------------------------
%	CHAPTER 4
%----------------------------------------------------------------------------------------

\chapterimage{head1.png} % Chapter heading image

\chapter{555}

\section{Overview}
In this section you'll learn about one of the most famous integrated circuit (IC) in use. Each year millions of 555 Timer ICs are manufactured and sold.
It's named 555 because there are three \SI{5}{\kilo\ohm} resistors inside the IC. And as the name suggest, it is a timer circuit. The timing interval is controlled by an external resistor/capacitor network. And by changing the values for the resistor and capacitor the timing duration can be easily varied.

Let's take a look at the pins of 555 Timer IC:
\begin{enumerate}
    \item \textbf{GND - Pin 1} Ground pin of the IC
    \item \textbf{VCC - Pin 8} Positive supply is connected to this pin, the voltage must be at least \SI{4.5}{\volt} and maximum \SI{15}{\volt}.
    \item \textbf{OUT - Pin 3} The output is either low (close to \SI{0}{\volt}) or high (close to VCC).
    \item \textbf{TRG - Pin 2} Trigger is active low, which means when the voltage on this pin drops below one-third of the supply voltage, the output of 555 goes high.
    \item \textbf{DIS - Pin 7} This pin is used to discharge an external capacitor that works in conjunction with a resistor to control the timing of the 555 IC.
    \item \textbf{THR - Pin 6} Threshold pin is used to monitor the voltage across the capacitor that's discharged by pin 7. When this voltage reaches two-third of the supply voltage, the output goes low.
    \item \textbf{CTRL - Pin 5} Control pin can be used to vary the voltage level at the inverting input of the threshold comparator. It is generally connected to ground via \SI{0.01}{\micro\farad} capacitor to eliminate any fluctuation on noise in the operation of the timer.
    \item \textbf{RST - Pin 4} Reset pin is active low, which means when this pin is momentarily grounded the 555 timer will reset it's state and will stop until it is triggered again.
\end{enumerate}

\begin{figure}[!h]
    \centering
    \begin{circuitikz}[scale = 1.5]
        \draw (0,0) to [short, *-] (4.5,0) {};
        \draw (4.5,0) node[npn, anchor=E, rotate=-90, color=red](npn1){};
        \draw (npn1.C) to [short, -o] (7,0){} node[above=1mm]{DIS} node[below=1mm]{7};
        \draw (0,0) node[ground]{};
        \draw (0,0) 
            to [R, l=\SI{5}{\kilo\ohm}, color=blue] (0,1.8)
            to [R, l=\SI{5}{\kilo\ohm}, color=blue] (0, 3.6)
            to [R, l=\SI{5}{\kilo\ohm}, color=blue] (0, 5.4);
        \draw (-0.2,5.4) -- node[anchor=south, color=red] {VCC} (0.2,5.4);
        \draw (1.2,1.8) node[op amp, color=red](opamp1){};
        \draw (1.2,3.6) node[op amp, color=red](opamp2){};
        \draw 
            (opamp1.+) to [short, -*] ++(-0.4,0){}
            (opamp2.-) to [short, -*] ++(-0.4,0){} 
                to [short, -o] ++(-1,0)
                node[above=0.2mm]{CTRL} node[below=0.2mm]{5};
        \draw
            (opamp1.-) to [crossing] ++(-0.8,0) to[short, -o] ++(-0.6,0) 
            node[above=0.2mm]{TRG} node[below=0.2mm]{2}
            (opamp2.+) to [crossing] ++(-0.8,0) to[short, -o] ++(-0.6,0)
            node[above=0.2mm]{THR} node[below=0.2mm]{6};
        \draw (3.5,2.7) node[flipflop sr-rst, external pins width=0] (SR1){};
        \draw 
            (SR1.pin 1) -| (opamp2.out)
            (SR1.pin 3) -| (opamp1.out)
            (SR1.up) -- ++(0,1) 
            to [short,-o] ++(0,1) 
            node[left=1mm]{\ctikztextnot{RST}} node[right=1mm]{4};
        \draw 
            let 
                \p1 = (npn1.B),
                \p2 = (SR1.pin 6)
            in 
                (SR1.pin 6) to [short, -*] (\x1, \y2){}
                (npn1.B) -- (\x1, \y2){}
                (\x1, \y2) to[inline not] ++(2,0) 
                to[short, -o] ++(0,0) node[above=1mm]{OUT} node[below=1mm]{3};
        %\draw (SR1.pin 6) to [short, -*] ++(1,0) coordinate(nin);
        %\draw (npn1.B) -- (nin);
        %\draw (nin) to[inline not] ++(2,0);
    \end{circuitikz}
    \caption{555 Timer Circuit}
    \label{fig:555_internal_circuit}
\end{figure}

Figure \ref{fig:555_symbol} shows the schematic symbol for 555 IC that we will use in this chapter's circuit examples.
\begin{figure}[!h]
    \centering
    \begin{circuitikz}[scale = 1.5]
        \TIMER555(0,0){1}
    \end{circuitikz}
    \caption{555 Timer Symbol}
    \label{fig:555_symbol}
\end{figure}

\section{555 : Operating Modes}
The 555 timer has 3 modes of operation and all of the upcoming activities utilizes one or more operation modes of 555.
In this section we will learn how to use different modes of operation of 555, after that we will build circuits using these modes.
\subsection{Astable Mode}
As the name suggests, in astable mode there is no stable state. The output continuously switches between high and low producing an
square wave. This circuit can be used for turning an LED on and off at regular intervals or act as a clock input for digital ICs
or control a motor by switching it on and off at regular time period.
\subsection{Monostable Mode}
Monostable means only one stable state. In this mode 555 has only one stable state and can produce a pulse of set duration as a 
response against a trigger. The output stays low (the stable state) as long as there is no trigger received by the 555. Once, a
trigger event happens, the output momentarily goes to high and then falls back to low after a set duration. This circuit can be 
used to provide a delay pulse, or turn on LED or motor or any mechanism for a fixed duration of time.
\subsection{Bistable Mode}
In Bistable mode the 555 has two stable states. When it receives a trigger input pulse, the output goes to high state and stays there 
until it receives a reset pulse, which makes the output fall back to low. This circuit is sometimes called as flip/flop also, because 
it can store the value of it's state for as long as the device is not reset or set.

\clearpage

\section{Lesson 11: 555 LED Flasher}
\subsection{Components Required}
\begin{enumerate}
    \item Breadboard Power Supply $\times$ 1
    \item 9V Battery $\times$ 1
    \item 9V Battery Connector $\times$ 1
    \item Breadboard $\times$ 1
    \item 555 IC $\times$ 1
    \item Red LED $\times$ 1
    \item \SI{220}{\ohm} $\times$ 1
    \item \SI{10}{\kilo\ohm} $\times$ 1
    \item \SI{100}{\kilo\ohm} $\times$ 1
    \item \SI{100}{\nano\farad} $\times$ 1
    \item \SI{10}{\micro\farad} $\times$ 1
    \item Male-Male jumper wire $\times$ 7
\end{enumerate}
\subsection{Circuit Picture}
\begin{figure}[!h]
    \centering
    \includegraphics[width=0.8\textwidth]{lesson_circuits/L11/lesson_11.png}
    \caption{555 LED flasher Breadboard Schematic}
    \label{fig:555_led_sch}
\end{figure}
\begin{figure}[!h]
    \centering
    \includegraphics[width=\textwidth]{lesson_circuits/L11/L11-A.png}
    \caption{555 LED flasher: LED OFF}
    \label{fig:555_led_obb}
\end{figure}
\begin{figure}[!h]
    \centering
    \includegraphics[width=\textwidth]{lesson_circuits/L11/L11-B.png}
    \caption{555 LED flasher: LED ON}
    \label{fig:555_led_obb1}
\end{figure}

\section{Lesson 12: 555 Dual LED Flasher}
\subsection{Components Required}
\begin{enumerate}
    \item Breadboard Power Supply $\times$ 1
    \item 9V Battery $\times$ 1
    \item 9V Battery Connector $\times$ 1
    \item Breadboard $\times$ 1
    \item 555 IC $\times$ 1
    \item Red LED $\times$ 1
    \item Blue LED $\times$ 1
    \item \SI{220}{\ohm} $\times$ 2
    \item \SI{10}{\kilo\ohm} $\times$ 1
    \item \SI{100}{\kilo\ohm} $\times$ 1
    \item \SI{100}{\nano\farad} $\times$ 1
    \item \SI{10}{\micro\farad} $\times$ 1
    \item Male-Male jumper wire $\times$ 7
\end{enumerate}
\subsection{Circuit Picture}
\begin{figure}[!h]
    \centering
    \includegraphics[width=0.8\textwidth]{lesson_circuits/L12/lesson_12.png}
    \caption{555 Dual LED flasher Breadboard Schematic}
    \label{fig:555_2led_sch}
\end{figure}
\begin{figure}[!h]
    \centering
    \includegraphics[width=\textwidth]{lesson_circuits/L12/L12-A.png}
    \caption{555 Dual LED flasher 1}
    \label{fig:555_2led_obb}
\end{figure}
\begin{figure}[!h]
    \centering
    \includegraphics[width=\textwidth]{lesson_circuits/L12/L12-B.png}
    \caption{555 LED flasher 2}
    \label{fig:555_2led_obb1}
\end{figure}

\section{Lesson 13: Fading LED using 555}
\subsection{Components Required}
\begin{enumerate}
    \item Breadboard Power Supply $\times$ 1
    \item 9V Battery $\times$ 1
    \item 9V Battery Connector $\times$ 1
    \item Breadboard $\times$ 1
    \item 555 IC $\times$ 1
    \item Red LED $\times$ 1
    \item 2N2222 $\times$ 1
    \item \SI{220}{\ohm} $\times$ 1
    \item \SI{10}{\kilo\ohm} $\times$ 1
    \item \SI{100}{\kilo\ohm} $\times$ 1
    \item \SI{1}{\Mohm} $\times$ 1
    \item \SI{100}{\nano\farad} $\times$ 1
    \item \SI{10}{\micro\farad} $\times$ 2
    \item Male-Male jumper wire $\times$ 11
\end{enumerate}
\subsection{Circuit Picture}
\begin{figure}[!h]
    \centering
    \includegraphics[width=0.8\textwidth]{lesson_circuits/L13/lesson_13.png}
    \caption{Fading LED using 555 Breadboard Schematic}
    \label{fig:555_fled_sch}
\end{figure}
\begin{figure}[!h]
    \centering
    \includegraphics[width=\textwidth]{lesson_circuits/L13/L13-A.png}
    \caption{LED fading 1}
    \label{fig:555_fled_obb}
\end{figure}
\begin{figure}[!h]
    \centering
    \includegraphics[width=\textwidth]{lesson_circuits/L13/L13-B.png}
    \caption{LED fading 2}
    \label{fig:555_fled_obb1}
\end{figure}

\section{Lesson 14: Bistable Button Flip/Flop using 555}
\subsection{Components Required}
\begin{enumerate}
    \item Breadboard Power Supply $\times$ 1
    \item 9V Battery $\times$ 1
    \item 9V Battery Connector $\times$ 1
    \item Breadboard $\times$ 1
    \item 555 IC $\times$ 1
    \item Red LED $\times$ 1
    \item Push Button $\times$ 2
    \item \SI{220}{\ohm} $\times$ 1
    \item \SI{10}{\kilo\ohm} $\times$ 2
    \item \SI{100}{\nano\farad} $\times$ 2
    \item Male-Male jumper wire $\times$ 9
\end{enumerate}
\subsection{Circuit Picture}
\begin{figure}[!h]
    \centering
    \includegraphics[width=0.8\textwidth]{lesson_circuits/L14/lesson_14.png}
    \caption{Bistable Button Flip/Flop using 555 Breadboard Schematic}
    \label{fig:555_ff_sch}
\end{figure}
\begin{figure}[!h]
    \centering
    \includegraphics[width=\textwidth]{lesson_circuits/L14/L14-A.png}
    \caption{FF Idle}
    \label{fig:555_ff_obb}
\end{figure}
\begin{figure}[!h]
    \centering
    \includegraphics[width=\textwidth]{lesson_circuits/L14/L14-B.png}
    \caption{FF SET Button Pressed}
    \label{fig:555_ff_obb1}
\end{figure}
\begin{figure}[!h]
    \centering
    \includegraphics[width=\textwidth]{lesson_circuits/L14/L14-C.png}
    \caption{FF Idle after leaving SET button}
    \label{fig:555_ff_obb2}
\end{figure}
\begin{figure}[!h]
    \centering
    \includegraphics[width=\textwidth]{lesson_circuits/L14/L14-D.png}
    \caption{FF RST Button Pressed}
    \label{fig:555_ff_obb3}
\end{figure}
\begin{figure}[!h]
    \centering
    \includegraphics[width=\textwidth]{lesson_circuits/L14/L14-E.png}
    \caption{FF Idle after leaving RST Button}
    \label{fig:555_ff_obb4}
\end{figure}
\section{Lesson 15: Toggle Switch with 555}
\subsection{Components Required}
\begin{enumerate}
    \item Breadboard Power Supply $\times$ 1
    \item 9V Battery $\times$ 1
    \item 9V Battery Connector $\times$ 1
    \item Breadboard $\times$ 1
    \item 555 IC $\times$ 1
    \item Red LED $\times$ 1
    \item Push Button $\times$ 1
    \item \SI{220}{\ohm} $\times$ 1
    \item \SI{10}{\kilo\ohm} $\times$ 2
    \item \SI{100}{\kilo\ohm} $\times$ 1
    \item \SI{100}{\nano\farad} $\times$ 2
    \item Male-Male jumper wire $\times$ 11
\end{enumerate}
\subsection{Circuit Picture}
\begin{figure}[!h]
    \centering
    \includegraphics[width=0.8\textwidth]{lesson_circuits/L15/lesson_15.png}
    \caption{Toggle Switch using 555 Breadboard Schematic}
    \label{fig:555_ts_sch}
\end{figure}
\begin{figure}[!h]
    \centering
    \includegraphics[width=\textwidth]{lesson_circuits/L15/L15-A.png}
    \caption{Idle}
    \label{fig:555_ts_obb}
\end{figure}
\begin{figure}[!h]
    \centering
    \includegraphics[width=\textwidth]{lesson_circuits/L15/L15-B.png}
    \caption{Button Pressed: LED turned ON}
    \label{fig:555_ts_obb1}
\end{figure}
\begin{figure}[!h]
    \centering
    \includegraphics[width=\textwidth]{lesson_circuits/L15/L15-C.png}
    \caption{Button released}
    \label{fig:555_ts_obb2}
\end{figure}
\begin{figure}[!h]
    \centering
    \includegraphics[width=\textwidth]{lesson_circuits/L15/L15-D.png}
    \caption{Button Pressed: LED turned OFF}
    \label{fig:555_ts_obb3}
\end{figure}
\begin{figure}[!h]
    \centering
    \includegraphics[width=\textwidth]{lesson_circuits/L15/L15-E.png}
    \caption{Button released}
    \label{fig:555_ts_obb4}
\end{figure}
\section{Lesson 16: Timer Delay using 555}
\subsection{Components Required}
\begin{enumerate}
    \item Breadboard Power Supply $\times$ 1
    \item 9V Battery $\times$ 1
    \item 9V Battery Connector $\times$ 1
    \item Breadboard $\times$ 1
    \item 555 IC $\times$ 1
    \item Red LED $\times$ 1
    \item Push Button $\times$ 1
    \item \SI{220}{\ohm} $\times$ 1
    \item \SI{10}{\kilo\ohm} $\times$ 1
    \item \SI{1}{\Mohm} $\times$ 1
    \item \SI{100}{\nano\farad} $\times$ 2
    \item \SI{100}{\nano\farad} $\times$ 2
    \item Male-Male jumper wire $\times$ 11
\end{enumerate}
\subsection{Circuit Picture}
\begin{figure}[!h]
    \centering
    \includegraphics[width=0.8\textwidth]{lesson_circuits/L15/lesson_15.png}
    \caption{Timer Delay using 555 Breadboard Schematic}
    \label{fig:555_timer_sch}
\end{figure}
\begin{figure}[!h]
    \centering
    \includegraphics[width=\textwidth]{lesson_circuits/L15/L15-A.png}
    \caption{Idle}
    \label{fig:555_timer_obb}
\end{figure}
\begin{figure}[!h]
    \centering
    \includegraphics[width=\textwidth]{lesson_circuits/L15/L15-B.png}
    \caption{Button Pressed}
    \label{fig:555_timer_obb1}
\end{figure}
\begin{figure}[!h]
    \centering
    \includegraphics[width=\textwidth]{lesson_circuits/L15/L15-C.png}
    \caption{Button released}
    \label{fig:555_timer_obb2}
\end{figure}
\begin{figure}[!h]
    \centering
    \includegraphics[width=\textwidth]{lesson_circuits/L15/L15-D.png}
    \caption{LED turned off after delay time}
    \label{fig:555_timer_obb3}
\end{figure}
\section{Lesson 17: Single Tone Buzzer with 555}
\subsection{Components Required}
\begin{enumerate}
    \item Breadboard Power Supply $\times$ 1
    \item 9V Battery $\times$ 1
    \item 9V Battery Connector $\times$ 1
    \item Breadboard $\times$ 1
    \item 555 IC $\times$ 1
    \item Active Buzzer $\times$ 1
    \item \SI{10}{\kilo\ohm} $\times$ 1
    \item \SI{100}{\kilo\ohm} $\times$ 1
    \item \SI{100}{\nano\farad} $\times$ 1
    \item \SI{10}{\micro\farad} $\times$ 1
    \item Male-Male jumper wire $\times$ 7
\end{enumerate}
\subsection{Circuit Picture}
\begin{figure}[!h]
    \centering
    \includegraphics[width=0.8\textwidth]{lesson_circuits/L17/lesson_17.png}
    \caption{Single tone Buzzer with 555 Breadboard Schematic}
    \label{fig:555_sbuz_sch}
\end{figure}
\begin{figure}[!h]
    \centering
    \includegraphics[width=\textwidth]{lesson_circuits/L17/L17-A.png}
    \caption{Single tone Buzzer}
    \label{fig:555_sbuz_obb}
\end{figure}
\section{Lesson 18: Short Beep}
\subsection{Components Required}
\begin{enumerate}
    \item Breadboard Power Supply $\times$ 1
    \item 9V Battery $\times$ 1
    \item 9V Battery Connector $\times$ 1
    \item Breadboard $\times$ 1
    \item 555 IC $\times$ 1
    \item Active Buzzer $\times$ 1
    \item 1N4007 Diode $\times$ 1
    \item \SI{10}{\kilo\ohm} $\times$ 1
    \item \SI{100}{\kilo\ohm} $\times$ 1
    \item \SI{1}{\Mohm} $\times$ 1
    \item \SI{100}{\nano\farad} $\times$ 1
    \item \SI{2.2}{\micro\farad} $\times$ 1
    \item Male-Male jumper wire $\times$ 10
\end{enumerate}
\subsection{Circuit Picture}
\begin{figure}[!h]
    \centering
    \includegraphics[width=0.8\textwidth]{lesson_circuits/L18/lesson_18.png}
    \caption{Short Beep using 555 Breadboard Schematic}
    \label{fig:555_sbeep_sch}
\end{figure}
\begin{figure}[!h]
    \centering
    \includegraphics[width=\textwidth]{lesson_circuits/L18/L18-A.png}
    \caption{Short Beep}
    \label{fig:555_sbeep_obb}
\end{figure}
\section{Lesson 19: Break Beam Detector using 555 and LDR}
\subsection{Components Required}
\begin{enumerate}
    \item Breadboard Power Supply $\times$ 1
    \item 9V Battery $\times$ 1
    \item 9V Battery Connector $\times$ 1
    \item Breadboard $\times$ 1
    \item 555 IC $\times$ 1
    \item Passive Buzzer $\times$ 1
    \item White LED $\times$ 1
    \item LDR $\times$ 1
    \item \SI{220}{\ohm} $\times$ 1
    \item \SI{10}{\kilo\ohm} $\times$ 3
    \item \SI{100}{\nano\farad} $\times$ 2
    \item \SI{10}{\micro\farad} $\times$ 1
    \item Male-Male jumper wire $\times$ 15
\end{enumerate}
\subsection{Circuit Picture}
\begin{figure}[!h]
    \centering
    \includegraphics[width=0.8\textwidth]{lesson_circuits/L19/lesson_19.png}
    \caption{Break Beam Detector using 555 and LDR Breadboard Schematic}
    \label{fig:555_bbdet_sch}
\end{figure}
\begin{figure}[!h]
    \centering
    \includegraphics[width=\textwidth]{lesson_circuits/L19/L19-A.png}
    \caption{No Obstacle: Buzzer OFF}
    \label{fig:555_bbdet_obb}
\end{figure}
\begin{figure}[!h]
    \centering
    \includegraphics[width=\textwidth]{lesson_circuits/L19/L19-B.png}
    \caption{Obstacle: Buzzer ON}
    \label{fig:555_bbdet_obb1}
\end{figure}
\section{Lesson 20: Light reactive buzzer using 555 and LDR}
\subsection{Components Required}
\begin{enumerate}
    \item Breadboard Power Supply $\times$ 1
    \item 9V Battery $\times$ 1
    \item 9V Battery Connector $\times$ 1
    \item Breadboard $\times$ 1
    \item 555 IC $\times$ 1
    \item Passive Buzzer $\times$ 1
    \item LDR $\times$ 1
    \item \SI{10}{\kilo\ohm} $\times$ 1
    \item \SI{100}{\nano\farad} $\times$ 2
    \item \SI{10}{\micro\farad} $\times$ 1
    \item Male-Male jumper wire $\times$ 9
\end{enumerate}
\subsection{Circuit Picture}
\begin{figure}[!h]
    \centering
    \includegraphics[width=0.8\textwidth]{lesson_circuits/L20/lesson_20.png}
    \caption{Light reactive Buzzer Breadboard Schematic}
    \label{fig:555_ldrbuzz_sch}
\end{figure}
\begin{figure}[!h]
    \centering
    \includegraphics[width=\textwidth]{lesson_circuits/L20/L20-A.png}
    \caption{Light reactive Buzzer}
    \label{fig:555_ldrbuzz_obb}
\end{figure}
\section{Lesson 21: Audio Tone/Siren}
\subsection{Components Required}
\begin{enumerate}
    \item Breadboard Power Supply $\times$ 1
    \item 9V Battery $\times$ 1
    \item 9V Battery Connector $\times$ 1
    \item Breadboard $\times$ 1
    \item 555 IC $\times$ 2
    \item Passive Buzzer $\times$ 1
    \item \SI{10}{\kilo\ohm} $\times$ 5
    \item \SI{100}{\kilo\ohm} $\times$ 1
    \item \SI{100}{\nano\farad} $\times$ 2
    \item \SI{10}{\micro\farad} $\times$ 2
    \item Male-Male jumper wire $\times$ 14
\end{enumerate}
\subsection{Circuit Picture}
\begin{figure}[!h]
    \centering
    \includegraphics[width=0.8\textwidth]{lesson_circuits/L21/lesson_21.png}
    \caption{Audio Tone/Siren Breadboard Schematic}
    \label{fig:555_audsi_sch}
\end{figure}
\begin{figure}[!h]
    \centering
    \includegraphics[width=\textwidth]{lesson_circuits/L21/L21-A.png}
    \caption{Audio Tone/Siren}
    \label{fig:555_audsi_obb}
\end{figure}
\section{Lesson 22: Traffic Light}
\subsection{Components Required}
\begin{enumerate}
    \item Breadboard Power Supply $\times$ 1
    \item 9V Battery $\times$ 1
    \item 9V Battery Connector $\times$ 1
    \item Breadboard $\times$ 1
    \item 555 IC $\times$ 2
    \item Red LED $\times$ 1
    \item Yellow LED $\times$ 1
    \item Green LED $\times$ 1
    \item 2N2222 $\times$ 1
    \item \SI{220}{\ohm} $\times$ 3
    \item \SI{330}{\ohm} $\times$ 3
    \item \SI{10}{\kilo\ohm} $\times$ 2
    \item \SI{1}{\Mohm} $\times$ 2
    \item \SI{100}{\nano\farad} $\times$ 2
    \item \SI{2.2}{\micro\farad} $\times$ 1
    \item \SI{10}{\micro\farad} $\times$ 2
    \item Male-Male jumper wire $\times$ 22
\end{enumerate}
\subsection{Circuit Picture}
\begin{figure}[!h]
    \centering
    \includegraphics[width=0.8\textwidth]{lesson_circuits/L22/lesson_22.png}
    \caption{Traffic Light Breadboard Schematic}
    \label{fig:555_trlight_sch}
\end{figure}
\begin{figure}[!h]
    \centering
    \includegraphics[width=\textwidth]{lesson_circuits/L22/L22-A.png}
    \caption{Green Light On}
    \label{fig:555_trlight_obb}
\end{figure}
\begin{figure}[!h]
    \centering
    \includegraphics[width=\textwidth]{lesson_circuits/L22/L22-C.png}
    \caption{Yellow Light On}
    \label{fig:555_trlight_obb1}
\end{figure}
\begin{figure}[!h]
    \centering
    \includegraphics[width=\textwidth]{lesson_circuits/L22/L22-B.png}
    \caption{Red Light On}
    \label{fig:555_trlight_obb2}
\end{figure}
\section{Lesson 23: Doorbell}
\subsection{Components Required}
\begin{enumerate}
    \item Breadboard Power Supply $\times$ 1
    \item 9V Battery $\times$ 1
    \item 9V Battery Connector $\times$ 1
    \item Breadboard $\times$ 1
    \item 555 IC $\times$ 2
    \item 2N2222 $\times$ 3
    \item Passive Buzzer $\times$ 1
    \item Push Button $\times$ 1
    \item \SI{1}{\kilo\ohm} $\times$ 3
    \item \SI{10}{\kilo\ohm} $\times$ 3
    \item \SI{1}{\Mohm} $\times$ 1
    \item \SI{100}{\nano\farad} $\times$ 2
    \item \SI{1}{\micro\farad} $\times$ 1
    \item Male-Male jumper wire $\times$ 23
\end{enumerate}
\subsection{Circuit Picture}
\begin{figure}[!h]
    \centering
    \includegraphics[width=0.8\textwidth]{lesson_circuits/L23/lesson_23.png}
    \caption{Doorbell Breadboard Schematic}
    \label{fig:555_doorbell_sch}
\end{figure}
\begin{figure}[!h]
    \centering
    \includegraphics[width=\textwidth]{lesson_circuits/L23/L23-A.png}
    \caption{Idle: Buzzer OFF}
    \label{fig:555_doorbell_obb}
\end{figure}
\begin{figure}[!h]
    \centering
    \includegraphics[width=\textwidth]{lesson_circuits/L23/L23-B.png}
    \caption{Button Pressed: Buzzer ON for sometime}
    \label{fig:555_doorbell_obb1}
\end{figure}
\section{Lesson 24: PWM Speed Controller}
\subsection{Components Required}
\begin{enumerate}
    \item Breadboard Power Supply $\times$ 1
    \item 9V Battery $\times$ 1
    \item 9V Battery Connector $\times$ 1
    \item Breadboard $\times$ 1
    \item 555 IC $\times$ 1
    \item 2N2222 $\times$ 1
    \item DC Motor $\times$ 1
    \item Propeller $\times$ 1
    \item 1N4007 Diode $\times$ 2
    \item \SI{1}{\kilo\ohm} $\times$ 1
    \item \SI{2}{\kilo\ohm} $\times$ 1
    \item \SI{10}{\kilo\ohm} $\times$ 1
    \item \SI{10}{\kilo\ohm} Potentiometer $\times$ 1
    \item \SI{100}{\nano\farad} $\times$ 2
    \item \SI{10}{\micro\farad} $\times$ 1
    \item Male-Male jumper wire $\times$ 11
\end{enumerate}
\subsection{Circuit Picture}
\begin{figure}[!h]
    \centering
    %\includegraphics[width=0.8\textwidth]{lesson_circuits/L24/lesson_24.png}
    \caption{PWM Speed Controller Breadboard Schematic}
    \label{fig:555_pwm_sch}
\end{figure}
\begin{figure}[!h]
    \centering
    \includegraphics[width=\textwidth]{lesson_circuits/L24/L24-C.png}
    \caption{PWM 0 Duty Cycle}
    \label{fig:555_pwm_obb}
\end{figure}
\begin{figure}[!h]
    \centering
    \includegraphics[width=\textwidth]{lesson_circuits/L24/L24-A.png}
    \caption{PWM half Duty Cycle}
    \label{fig:555_pwm_obb1}
\end{figure}
\begin{figure}[!h]
    \centering
    \includegraphics[width=\textwidth]{lesson_circuits/L24/L24-B.png}
    \caption{PWM full Duty Cycle}
    \label{fig:555_pwm_obb2}
\end{figure}
\section{Lesson 25: 555 RGB Flasher}
\subsection{Components Required}
\begin{enumerate}
    \item Breadboard Power Supply $\times$ 1
    \item 9V Battery $\times$ 1
    \item 9V Battery Connector $\times$ 1
    \item Breadboard $\times$ 1
    \item 555 IC $\times$ 3
    \item RGB LED $\times$ 1
    \item 2N2222 $\times$ 3
    \item \SI{220}{\ohm} $\times$ 3
    \item \SI{10}{\kilo\ohm} $\times$ 3
    \item \SI{100}{\kilo\ohm} $\times$ 3
    \item \SI{1}{\Mohm} $\times$ 3
    \item \SI{100}{\nano\farad} $\times$ 3
    \item \SI{2.2}{\micro\farad} $\times$ 1
    \item \SI{4.7}{\micro\farad} $\times$ 1
    \item \SI{10}{\micro\farad} $\times$ 4
    \item Male-Male jumper wire $\times$ 30
\end{enumerate}
\subsection{Circuit Picture}
\begin{figure}[!h]
    \centering
    \includegraphics[width=0.8\textwidth]{lesson_circuits/L25/lesson_25.png}
    \caption{RGB LED Flasher Breadboard Schematic}
    \label{fig:555_rgb_sch}
\end{figure}
\begin{figure}[!h]
    \centering
    \includegraphics[width=\textwidth]{lesson_circuits/L25/L25-A.png}
    \caption{RGB Flasher}
    \label{fig:555_rgb_obb}
\end{figure}
\begin{figure}[!h]
    \centering
    \includegraphics[width=\textwidth]{lesson_circuits/L25/L25-B.png}
    \caption{RGB Flasher}
    \label{fig:555_rgb_obb1}
\end{figure}